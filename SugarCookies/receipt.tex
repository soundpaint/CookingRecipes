\NeedsTeXFormat{LaTeX2e}
\documentclass[12pt,utf8,a4paper]{article}
\usepackage[utf8]{inputenc}
\usepackage[T1]{fontenc}
\usepackage[pdftex]{graphicx}
\usepackage{textcomp}

\setlength{\parindent}{0pt}
\setlength{\topmargin}{0pt}
\setlength{\textheight}{26cm}

\begin{document}

\pagestyle{empty}

\begin{figure}[htb]
  \centering \leavevmode
  \includegraphics[angle=0,width=0.95\textwidth]{header}
\end{figure}

\begin{minipage}[t]{0.66\textwidth}
  \begin{itemize}
  \item 500g Mehl
  \item 200g Zucker
  \item 2 gestrichene Teelöffel Backpulver
  \item 2 Eier
  \item 1 Päckchen Vanillezucker
  \item Zitronenschalenarome / Bittermandelaroma
  \item 250g Butter (weich, in Scheibchen schneiden)
  \end{itemize}
\end{minipage}
\hfill
\begin{minipage}[t]{0.30\textwidth}
  \underline{Zeit:}\\
  Bei 180\textdegree{}C etwa 10 Min\\
  untere Hälfte des\\
  Backofens
\end{minipage}

\vspace{1.5em}
Küchenmaschine mit \underline{Knethaken auf Stufe 1}:

\begin{itemize}
\item Erst Mehl, Zucker, Backpulver,
  Vanillezucker gründlich
  vermischen,
\item Eier und Gewürze dazu,
\item nach und nach Butterstückchen
  reinfüllen,
\item solange kneten lassen, bis
  sich der Teig vom Rand her zu
  lösen beginnt,
\item Teig rausnehmen, in 2 Rollen
  formen und\\30 Minuten in Ruhe
  lassen, dann weiterverarbeiten:
\end{itemize}

\begin{description}
\item[Fleischwolf / Spritzgebäck:]
  Teig \underline{nicht} kühlen
\item[Anrollteig:] Erst gut 30
  Minuten in den Kühlschrank
  (Ausrollen geht auch mittels einer
  \underline{glatten Flasche})
\item[Marmeladeplätzchen:] Kleine
  Teigbällchen (nussgroß)\\in der
  Mitte eindrücken, dann Marmelade
  reinfüllen\\(Erdbeer- oder
  Himbeermarmelade)
\item[Sehr zu empfehlen:]
  \underline{Backpapierbögen} (spart
  das Säubern vom Backblech)
\end{description}

\end{document}
