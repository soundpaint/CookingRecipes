\NeedsTeXFormat{LaTeX2e}
\PassOptionsToPackage{svgnames}{xcolor}
\documentclass[12pt,utf8,a4paper]{article}
\usepackage[utf8]{inputenc}
\usepackage[T1]{fontenc}
\usepackage[pdftex]{graphicx}
\usepackage{textcomp}
\usepackage{yfonts}
\usepackage[object=vectorian]{pgfornament}
\usepackage{tikz}

\setlength{\parindent}{0pt}
\setlength{\topmargin}{0pt}
\setlength{\textheight}{26cm}

\newcommand{\ornament}[3]{
  \nointerlineskip
  \vspace{.5\baselineskip}
  \hspace{\fill}
  {
    \color{#1}
    \resizebox{0.5\linewidth}{#3}
              {
                \begin{tikzpicture}
                  \node (C) at (0,0) {};
                  \node (D) at (9,0) {};
                  \path (C) to [ornament=#2] (D);
                \end{tikzpicture}
              }
  }
  \hspace{\fill}
  \par\nointerlineskip
  \vspace{.5\baselineskip}
}

\begin{document}

\pagestyle{empty}

\begin{center}
  {\textswab{\Huge Pfirsich-Aprikosen-Cremetorte}}
  {\LARGE à la \textswab{Jürgen}}\\
  \ornament{orange}{84}{2ex}

  \vspace{4em}

\begin{minipage}[t]{0.66\textwidth}
  {\center\LARGE\textswab{Zutaten}\\}
  \begin{itemize}
  \item[*] 1 Wiener Boden (hell)
  \item[*] 1 Dose (850ml) halbe gezuckerte Pfirsiche
  \item[*] 1 kleines Glas (250g) pürierte Aprikosenmarmelade
  \item[*] 1 Pkg.\ Kaltschale (für 500ml
    Flüssigkeit),\\ Geschmacksrichtung Pfirsich-Maracuja
  \item[*] 300ml gekühlte Milch, 3,5\% Fett
  \item[*] 2 Becher á 200ml gekühlte süße Sahne
  \item[*] ggf. 1 Pkg. Sahnesteif (für 200ml)
  \item[*] 10g dunkles, rohes Kakaopulver
  \item[*] 12× Schokoladendekor-Element
  \end{itemize}
  \ornament{brown}{85}{1ex}
  \vspace{2em}

  {\center\LARGE\textswab{Hilfs:mittel}\\}
  \begin{itemize}
  \item[*] Rührmixer
  \item[*] Messbecher, 1l
  \item[*] Runde Kuchenbackform
  \item[*] Messer, Schneidbrett
  \item[*] Esslöffel
  \end{itemize}
  \ornament{brown}{85}{1ex}
\end{minipage}

\end{center}

\newpage
{\center\LARGE\textswab{Zubereitung}\\}
\begin{itemize}
\item[*] Unteren Teil des Wiener Bodens in Kuchenbackform plazieren.
\item[*] Aus der Dose Aprikosen 4 wohlgeformte, etwa gleich große
  Hälften beiseite legen.
\item[*] Die restlichen Aprikosenhälften in kleine Stücke schneiden
  und auf dem unteren Teil des Bodens verteilen.  Dabei am Rand etwa
  1cm Abstand zur Backform halten.
\item[*] Mit der Marmelade gleichmäßig über die Aprikosenstückchen
  verteilen und auch dabei etwas Platz am Rand lassen.
\item[*] Den mittleren Teil des Bodens darauf setzen.
\item[*] 1 Becher Sahne und die Milch in den Messbecher füllen.  Das
  Kalt\-scha\-len\-pulver mit dem Löffel kurz einrühren und gut
  umrühren.  Den Messbecher für ca. 10 Minuten im Kühlschrank ruhen
  lassen, bis die Fruchtstückchen des Kaltschalenpulvers ein wenig
  aufgeweicht sind.
\item[*] Den Inhalt des Messbechers mit dem Rührmixer zu einer Creme
  aufschlagen.
\item[*] Die Creme auf dem mittleren Boden gleichmäßig bis an den Rand
  der Backform verteilen.  Den Messbecher anschließend \textit{nicht}
  ausspülen, sondern ein wenig der Creme (insgesamt ca. 1 Esslöffel)
  im Messbecher belassen.
\item[*] Den oberen Teil des Bodens aufsetzen.
\item[*] In den nicht ausgespülten Messbecher die übrige Sahne (200ml)
  füllen und mit dem Rührmixer steif aufschlagen, ggf. mit Sahnesteif.
  Dabei die verbliebenen Reste der Creme mit einrühren.
\item[*] Die Sahne auf dem oberen Teil des Bodens verteilen.
\item[*] Die übrigen 4 Pfirsichhälften in je 3 etwa gleich große
  sichelförmige Stücke teilen und in einem kleinen Kreis um den
  Mittelpunkt herum auf der Sahne plazieren.
\item[*] Das Kakaopulver über der Torte gleichmäßig streuen,
  z.B. durch Zerreiben kleiner Mengen zwischen Daumen und Zeigefinger.
\item[*] Die 12 Schokoladendekor-Elemente an dem Kreis aus
  Pfirsichstücken außen gleichmäßig anfügen.
\item[*] Die Torte mindestens 1 Stunde im Kühlschrank kalt stellen.
\item[*] Torte aufschneiden und servieren.
\end{itemize}

\ornament{DarkGreen}{88}{2ex}

\end{document}
